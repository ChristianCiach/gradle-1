\chapter{The Gradle Wrapper} % (fold)
\label{cha:the_gradle_wrapper}
If you want to use Gradle for building your software you might come across to problems. 
\begin{itemize}
	\item How to use Gradle with a continuos integration server where Gradle is not installed.
	\item How can I spare my users to install Gradle for building my project.
\end{itemize}
These might be no issues if you have control over your environment. But if you have no admin right for the continuos integration server or if you provide an open source project the problems described above arise. The simple and good news is, that Gradle provides a solution for this via its \emph{Gradle Wrapper}.

Gradle provides a Wrapper task. You can create such a task in your build script.

\begin{Verbatim}
createTask('wrapper', type: Wrapper).configure {
    gradleVersion = '0.1'
}	
\end{Verbatim}
This task is meant to be explicitly executed. After such an execution you find the following new files in your project folder.
\begin{Verbatim}
project-root
  - gradle-wrapper
     - gradle-wrapper.jar
  - gradlew
  - gradlew.bat
\end{Verbatim}
\noindent
All these files should be submitted to your version control system. The \texttt{gradlew} commands can be used \emph{exactly} the same way as the \texttt{gradle} commands. If you run Gradle with \texttt{gradlew}, Gradle looks if the directory \texttt{\emph{project-root}/gradle-wrapper/gradle-dist/gradle-0.1} exists. If it exists, the \texttt{gradle} command of this distribution is called with all the arguments passed to the \texttt{gradlew} command.
If the distribution directory does not exists, Gradle looks if the file \texttt{\emph{project-root}/gradle-wrapper/gradle-dist/gradle-0.1.zip} does exists. If so, the zip is unpacked and \texttt{gradle} is called. If the distribution zip does not exists, Gradle tries to download it. You can specify the download url root via the \texttt{urlRoot} property of the Wrapper task. If you don't specify it, Gradle uses \texttt{http://dist.codehaus.org/gradle} as its value. 

If you don't want any download to happen when your project is build via \texttt{gradlew}, simply add the Gradle distribution zip to your version control at the location described above. 

\section{Unix file permissions} % (fold)
\label{sec:unix_file_permissions}
The Wrapper task adds file permissions to allow execution for the gradlew UNIX command it creates. Subversion preserves this file permission. We are not sure how other version control systems deal with this. What should always work is to execute \texttt{sh gradlew}. 

% section unix_file_permissions (end)

\section{Environment variable} % (fold)
\label{sec:environment_variable}
There are rather exotic use cases like the continuos integration server went down during unzipping the Gradle distribution. As the distribution directory exists \texttt{gradlew} delegates to it but the distribution is corrupt. Or the zip-distribution was not properly downloaded. When you have no admin right for the continuous integration server to remove the corrupt files, Gradle offers an solution via environment variables.

\begin{tabularx}{\textwidth}{cX} 
	\textbf{Variable Name} & \centerline{\textbf{Meaning}}\\
    GRADLE\_WRAPPER\_ALWAYS\_UNPACK & If set to \texttt{true}, the distribution directory gets always deleted when \texttt{gradlew} is run and the distribution zip is freshly unpacked. If the zip is not there, Gradle tries to download it. \\
	GRADLE\_WRAPPER\_ALWAYS\_DOWNLOAD & If set to \texttt{true}, the distribution directory and the distribution zip gets always deleted when \texttt{gradlew} is run and the distribution zip is freshly downloaded. \\
\end{tabularx}
% section environment_variable (end)
% chapter the_gradle_wrapper (end)