%!TEX root = master.tex
\chapter{The Gradle Wrapper} % (fold)
\label{cha:the_gradle_wrapper}
Gradle is a new tool. You can't expect it to be installed on machines beyond your sphere of influence. An example are continuous integration server where Gradle is not installed and you have no admin rights for the machine. Or what if you provide an open source project and you want to make it as easy as possible for your users to build it.

There is a simple and good news. Gradle provides a solution for this. It ships with a \emph{Wrapper} task.\footnote{Gradle itself is continuously built by Bamboo and Teamcity via this wrapper. See \url{http://gradle.org/ci-server.html}} You can create such a task in your build script.
\begin{Verbatim}
createTask('wrapper', type: Wrapper).configure {
    gradleVersion = '0.1'
}	
\end{Verbatim}
This task is usually explicitly executed (for example after a switch to a new version of Gradle). After such an execution you find the following new or updated files in your project folder.
\begin{Verbatim}
project-root
  - gradle-wrapper
     - gradle-wrapper.jar
  - gradlew
  - gradlew.bat
\end{Verbatim}
\noindent
All these files should be submitted to your version control system. The \texttt{gradlew} commands can be used \emph{exactly} the same way as the \texttt{gradle} commands. If you run Gradle with \texttt{gradlew}, Gradle looks if the directory \texttt{\emph{project-root}/gradle-wrapper/gradle-dist/gradle-0.1} exists. If it exists, the \texttt{gradle} command of this distribution is called with all the arguments passed to the \texttt{gradlew} command.
If the distribution directory does not exists, Gradle looks if the file \texttt{\emph{project-root}/gradle-wrapper/gradle-dist/gradle-0.1.zip} does exists. If so, the zip is unpacked and \texttt{gradle} is called. If the distribution zip does not exists, Gradle tries to download it. You can specify the download url root via the \texttt{urlRoot} property of the Wrapper task. If you don't specify it, Gradle uses \texttt{http://dist.codehaus.org/gradle} as its value. 

If you don't want any download to happen when your project is build via \texttt{gradlew}, simply add the Gradle distribution zip to your version control at the location described above. 

\section{Unix file permissions} % (fold)
\label{sec:unix_file_permissions}
The Wrapper task adds appropriate file permissions to allow the execution for the gradlew *NIX command. Subversion preserves this file permission. We are not sure how other version control systems deal with this. What should always work is to execute \texttt{sh gradlew}. 

% section unix_file_permissions (end)

\section{Environment variable} % (fold)
\label{sec:environment_variable}
Some rather exotic use cases might occur when working with the Gradle Wrapper. For example the continuos integration server goes down during unzipping the Gradle distribution. As the distribution directory exists \texttt{gradlew} delegates to it but the distribution is corrupt. Or the zip-distribution was not properly downloaded. When you have no admin right on the continuous integration server to remove the corrupt files, Gradle offers a solution via environment variables.

\begin{tabularx}{\textwidth}{cX} 
	\textbf{Variable Name} & \centerline{\textbf{Meaning}}\\
    GRADLE\_WRAPPER\_ALWAYS\_UNPACK & If set to \texttt{true}, the distribution directory gets always deleted when \texttt{gradlew} is run and the distribution zip is freshly unpacked. If the zip is not there, Gradle tries to download it. \\
	GRADLE\_WRAPPER\_ALWAYS\_DOWNLOAD & If set to \texttt{true}, the distribution directory and the distribution zip gets always deleted when \texttt{gradlew} is run and the distribution zip is freshly downloaded. \\
\end{tabularx}
% section environment_variable (end)
% chapter the_gradle_wrapper (end)