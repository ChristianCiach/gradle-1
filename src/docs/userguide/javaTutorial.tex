%!TEX root = master.tex
\chapter{The Java Projects Appetizer} % (fold)
\label{cha:the_java_projects_tutorial}
We have in-depth coverage with many examples about the Java plugin, the dependency management and multi-project builds in the later chapters. In this chapter we want to give you just a first idea.

\section{Examples} % (fold)
\label{sec:examples}
Provided your build script contains the single line
\begin{Verbatim}
usePlugin('java')
\end{Verbatim}
Executing \texttt{gradle libs} will compile, test and jar your code. If you specify a remote repository, executing \texttt{gradle uploadLibs} will do additionally an upload of your jar to a remote repository. 
Builds have usually more requirements. Let's look at a typical multi-project build.

\begin{Verbatim}[frame=single,label=Project Tree]
D- ultimateApp
  D- api
  D- webservice
  D- shared 
\end{Verbatim}
We have three projects. \texttt{api} is shipped to the client to provide them a Java client for your XML webservice. \texttt{webservice} is a webapp which returns XML. \texttt{shared} is code used both by \texttt{api} and \texttt{webservice}. Let's look at the Gradle build scripts of those projects.

\begin{Verbatim}[frame=single,label=ultimateApp]
subprojects {
    manifest.mainAttributes([	       
	    'Implementation-Title': 'Gradle',
		'Implementation-Version': '0.1'
	])
	dependencies {
		compile "commons-lang:commons-lang:3.1"
		test "junit:junit:4.4"
	}
	sourceCompatibility = 1.5
	targetCompatibility = 1.5
	test {
	    include '**/*Test.class'
	    exclude '**/Abstract*'
	}
}
\end{Verbatim}
The commons stuff for all Java projects we do define in the root project. Not by inheritance but via \emph{Configuration Injection}. The root project is like a container and \texttt{subprojects} iterates over the elements of this container and injects the specified configuration. This way we can easily define the manifest content for all archives.

\begin{Verbatim}[frame=single,label=api]
dependencies {
	compile "commons-httpclient:commons-httpclient:3.1", project(":shared")
}

dists {
	zip() {
		files(dependencies.resolveClasspath("runtime")) // add dependencies to zip
		fileset(dir: "path/distributionFiles")
	}
}
\end{Verbatim}
In the \texttt{api} build script we add further dependencies. One dependency are the jars of the \texttt{shared} project. Due to this dependency \texttt{shared} gets now always build before \texttt{api}. We also add a distribution, that gets shipped to the client.
% section examples (end)



% chapter the_java_projects_tutorial (end)