%!TEX root = master.tex
\chapter{Command line} % (fold)
\label{cha:command_line}

\begin{tabular}{l|l|l}
\textbf{Option}(-) & \textbf{Long Option}(--) & \textbf{Meaning}\\ \hline
-?, -h  & -           & Shows this help message\\
B & bootstrap-debug & Specify a text to be logged at the beginning (used internally by Gradle's bootstrap class.)\\
D & systemprop       & Sets a system property of the JVM (e.g. -Dmyprop=myvalue).\\ 
I & noImports        & Disable usage of default imports for build script files.\\
P & projectprop  & Sets a project property of the root project (e.g. -Pmyprop=myvalue).\\
S & - & Don't trigger a System.exit(0) for normal termination. Used internally by Gradle's unit tests.\\
b & buildfile        & Specifies the build file name (also for subprojects). Defaults to build.gradle.\\
c & settingsfile     & Specifies the settings file name. Defaults to settings.gradle.\\
d & debug            & Log in debug mode (includes normal stacktrace).\\
e & embedded         & Specify an embedded build script.\\
f & full-stacktrace  & Print out the full (very verbose) stacktrace for any exceptions.\\
g & gradle-user-home & Specifies the gradle user home dir.\\
i & ivy-quiet        & Set Ivy log level to quiet.\\
j & ivy-debug        & Set Ivy log level to debug (very verbose).\\
l & plugin-properties-file & Specifies the plugin properties file.\\
p & project-dir       & Specifies the start dir for Gradle. Defaults to current dir.\\
q & quiet            & Log erros only.\\
r & rebuild-cache    & Rebuild the cache of compiled build scripts.\\
s & stacktrace       & Print out the stacktrace also for user exceptions (e.g. compile error).\\
t & tasks            & Show list of all available tasks and there dependencies.\\
u & no-search-upwards  & Don't search in parent folders for a settings.gradle file.\\
v & version          & Prints version info.\\
x & cache-off        & No caching of compiled build scripts.\\

\end{tabular}
\\
 
\noindent The same information is printed to the console when you execute \texttt{gradle -h}.\\
% chapter command_line (end)