%!TEX root = master.tex
\chapter{Command line} % (fold)
\label{cha:command_line}

\begin{tabular}{ccccc}
\hline
-D,--prop &  &  &  & Set system property of the JVM.\\
\hline
-P,--projectProperty &  Set project property of the root project.\\
-S,--noJvmTermination & Don't trigger a System.exit(0) for normal\\
termination. Useful for testing.\\
-b,--buildfile &  &    Use this build file name (also for subprojects)\\
-d,--debug &  &  &    Log in debug mode (includes normal stacktrace)\\
-e,--embedded &  &  & Use an embedded build script.\\
-f,--fullStacktrace &   Print out the full (very verbose) stacktrace.\\
-g,--gradleUserHome &   The user specific gradle dir.\\
-h,--help &  &  &  & usage information\\
-i,--depInfo &  &  &  info output from dependency management\\
-j,--depDebug &  &  & debug output from dependency management\\
-l,--pluginProperties & Name of the file with the plugin properties.\\
-n,--nonRecursive &  & Don't execute the tasks for the childprojects of\\
the current project\\
-p,--projectDir &  &   Use this dir instead of the current dir as the\\
project dir.\\
-q,--quiet &  &  &    Log in quiet mode.\\
-s,--stacktrace &  &   Print out the stacktrace.\\
-t,--tasks &  &  &    Show list of tasks.\\
-u,--noSearchUpwards &  Don't search in parent folders for\\
gradlesettings file.\\
-v,--version &  &  &  Prints put version info.\\
\hline
\end{tabular}



% chapter command_line (end)