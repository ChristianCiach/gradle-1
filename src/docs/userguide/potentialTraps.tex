%!TEX root = userguide.tex
\chapter{Potential Traps} % (fold)
\label{cha:potential_traps}
\section{Groovy Script Variables} % (fold)
\label{sec:groovy_script_variables}
For Gradle users it is important to understand how Groovy deals with script variables. Groovy has to type of script variables. One with a local scope and one with a script wide scope.
\codeInput{../../samples/userguide/tutorial/scope.groovy}
\outputInputTutorial{scope}
Variables which are declared with a type modifier are visible within closures but not visible within methods. This is a heavily discussed behavior in the Groovy community.\footnote{One of those discussions can be found here: \url{http://www.nabble.com/script-scoping-question-td16034724.html}}
% subsection groovy_script_variables (end)

\section{Configuration and Execution Phase} % (fold)
\label{sec:configuration_and_execution_phase}
It is important to keep in mind that Gradle has a distinct configuration and execution phase (see chapter \ref{cha:the_build_lifecycle}). 
\codeInput{../../samples/userguide/tutorial/mkdirTrap/build.gradle}
\outputInputTutorial{mkdirTrap}
As the creation of the directory happens during the configuration phase, the \texttt{clean} task removes the directory during the execution phase.
% chapter potential_traps (end)