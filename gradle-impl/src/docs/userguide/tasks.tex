%!TEX root = userguide.tex
\chapter{More about Tasks} % (fold)
\label{cha:more_about_tasks}
In the introductory tutorial (chapter \ref{cha:gradle_core_tutorial}) you have learned how to create simple task. You have also learned how to add additional behavior to those tasks later on. And you have learned how to create dependencies between tasks. This was all about simple tasks. But Gradle takes the concept of tasks further. Gradle supports enhanced tasks. Tasks which have there own properties and methods. This is really different to what you are used from Ant targets. Such enhanced tasks are either provided by you or are provided by Gradle.

\section{Configuring Tasks} % (fold)
\label{sec:configuring_tasks}
As an example, let's look at the Resources task provided by Gradle. To create such a Resources task for your build, you may declare in your build script\footnote{If you use the JavaPlugin, this task is automatically created and added to your project}:
\begin{Verbatim}
createTask('resources', type: Resources)
\end{Verbatim}
The resources task provides an API to configure it. See \href{\API tasks/Resources.html}{Resources Task API}. If you create the resources task like this, it has no default behavior.\footnote{This is different when added by the JavaPlugin}. We want now use the resources task to learn about different ways to configure it (the result is always the same for our examples).
\begin{Verbatim}
Resources resources = createTask('resources', type: Resources)
resources.from(file('resources'))
resources.to(file('target'))
resources.includes('**/*.txt', '**/*.xml', '**/*.properties')	
\end{Verbatim}
This is similar to the way we would normally configure objects in Java. You have to repeat the context (resources) in the configuration statement every time. This is a redundancy and not very nice to read.

There is a more convenient way of doing this.
\begin{Verbatim}
Resources resources = createTask('resources', type: Resources)
resources.from(file('resources')).to(file('target')).
	includes('**/*.txt', '**/*.xml', '**/*.properties')	
\end{Verbatim}
You might know this approach from the Hibernates Criteria Query API or JMock. Of course the API of a task has to support this. The from, to and includes methods all return an instance of the resources object. All of Gradles build-in tasks usually support this configuration style.

But there is yet another way of configuring the resources task. It also preserves the context and it possibly offers the best readability. It is usually our favorite.
\begin{Verbatim}
createTask('resources', type: Resources)
resources {
   from(file('resources'))
   to(file('target'))
   includes('**/*.txt', '**/*.xml', '**/*.properties')
}
\end{Verbatim}
This works for \emph{any} task. Line 2 of the example is just a shortcut for \texttt{task("resources")}. It is important to note that if you pass a closure to the \texttt{task} method, this closure is applied for \emph{configuring} the task. There is a slightly different ways of doing this.
\begin{Verbatim}
Resources resources = createTask('resources', type: Resources).configure {
   from(file('source'))
   to(file('target'))
   includes('**/*.txt', '**/*.xml', '**/*.properties')
}	
\end{Verbatim}
If you pass a closure to the \texttt{createTask()} method, this closure gets added as an \emph{action} to the task. Every task has a configure method, which you can pass a closure for configuring the task. The above example works, because the \texttt{createTask} method returns the task object. Gradle uses this style for configuring objects in many places, not just for tasks.
% section configuring_tasks (end)

\section{Adding Dependencies to Tasks}
\label{sec:adding_dependencies_to_tasks}
There are several ways you can define the dependencies of a task. In section \ref{sec:task_dependencies} you were
introduced to defining dependencies using task names. Task names can refer to tasks in the same project as the task, or
to tasks in other projects. To refer to a task in another project, you prefix the name of the task with the path of the
project it belongs to. Below is an example which adds a dependency from \texttt{projectA:taskX} to
\texttt{projectB:taskY}:
\codeInput{../../samples/userguide/tasks/addDependencyUsingPath/build.gradle}
\outputInputTutorial{addDependencyUsingPath}

Instead of using a task name, you can define a dependency using a \texttt{Task} object, as shown in this example:
\codeInput{../../samples/userguide/tasks/addDependencyUsingTask/build.gradle}
\outputInputTutorial{addDependencyUsingTask}

For more advanced uses, you can define a task dependency using a closure. When evaluated, the closure is passed the task
whose dependencies are being calculated. The closure should return a single \texttt{Task} or collection of \texttt{Task}
objects, which are then treated as dependencies of the task.  The following example adds a dependency from \texttt{taskX}
to all the tasks in the project whose name starts with \texttt{lib}:
\codeInput{../../samples/userguide/tasks/addDependencyUsingClosure/build.gradle}
\outputInputTutorial{addDependencyUsingClosure}

For more information about task dependencies, see the \href{\API Task.html}{org.gradle.api.Task} API.

% section adding_dependencies_to_tasks (end)

\section{Replacing Tasks}
Sometimes you want to replace a task. For example if you want to exchange a task added by the Java Plugin with a custom task of a different type. You can achieve this with:
\codeInput{../../samples/userguide/tutorial/replaceTask/build.gradle}
\outputInputTutorial{replaceTask}
Here we replace a task of type \texttt{Resources} with a simple task. When creating the simple task, you have to set the \texttt{overwrite} property to true. Otherwise Gradle throws an exception, saying that a task with such a name already exists.

\section{Summary} % (fold)
\label{sec:the_idea_behind_gradle_tasks}
If you are coming from Ant, such an enhanced Gradle task as \emph{Resources} looks like a mixture between an Ant target and an Ant task. And this is actually the case. The separation that Ant does between tasks and targets is not done by Gradle. The simple Gradle tasks are like Ant's targets and the enhanced Gradle tasks also include the Ant task aspects. All of Gradles tasks share a common API and you can create dependencies between them. Such a task might be nicer to configure than an Ant task. It makes full use of the type system, is more expressive and easier to maintain.

% You might be wondering why the Resources task you have created above is not initialized with any default values. After all Gradle claims to offer build-by-convention. But which default values? The one we use for Java Projects? Or the one we use for other project types? What we have done above is to use a Gradle task in a stand alone fashion. It is open for anything. How Gradles tasks are integrated into a build-by-convention mechanism is what we are going to show in the next chapter.

% section the_idea_behind_gradle_tasks (end)

% chapter more_about_tasks (end)